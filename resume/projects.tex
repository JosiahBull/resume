\cvsection{Relevant Projects}

\begin{cventries}
  \cventry
    {\href{https://github.com/JosiahBull/name-picker}{\faLink \space https://github.com/JosiahBull/name-picker}}
    {Name Picker}
    {}
    {2025}
    {
      \begin{cvitems}
        \item {A fullstack \textbf{Fullstack} \textbf{React} application built on top of \textbf{Vite} and \textbf{Supabase}, styling was done with \textbf{Material UI}.}
        \item {Optimised bundle sizes to 9kb for initial load, with total page size at `400kb` uncompressed.'}
        \item {Deployed to \textbf{Cloudflare Pages} via an custom CI pipeline and release process.}
      \end{cvitems}
    }

  \cventry
    {\href{https://github.com/JosiahBull/minilate}{\faLink \space https://github.com/JosiahBull/minilate}}
    {Minilate}
    {}
    {2025}
    {
      \begin{cvitems}
        \item {A \textbf{Rust} templating library based on \textbf{Handlebars} with a focus on small binary size, compile-time parsing, and execution speed.}
        \item {Faster by up to 30\% than other Rust templating libraries, with a significantly smaller binary size.}
      \end{cvitems}
    }

  \cventry
    {\href{https://github.com/JosiahBull/anyhow-tracing}{\faLink \space https://github.com/JosiahBull/anyhow-tracing}}
    {Anyhow Tracing}
    {}
    {2025}
    {
      \begin{cvitems}
        % TODO: add link to the github repo for `anyhow` here.
        \item {Created an iteration on the popular \textbf{Rust} error handling library `anyhow` with built in support for named tracing fields.}
        \item {Integrated library at Partly allowing \textbf{Sentry} alerts to contain more context and reduce time-to-debug.}
      \end{cvitems}
    }

  \cventry
    {\href{https://github.com/JosiahBull/human-centric-ids-rs}{\faLink \space https://github.com/JosiahBull/human-centric-ids-rs}}
    {Human Centric Ids}
    {}
    {2025}
    {
      \begin{cvitems}
        \item {An optimised low-level id-generation library, focused on producing inoffensive and legible ids for humans.}
        \item {Delivered a talk at a local RustConf about this library, wrote a small webserver for the demo using \textbf{Rocket}.}
      \end{cvitems}
    }

  \cventry
    {\href{https://github.com/JosiahBull/send}{\faLink \space https://github.com/JosiahBull/send}}
    {Send}
    {}
    {2024}
    {
      \begin{cvitems}
        \item {A \textbf{Rust} library for sharing files. Built to learn about \textbf{Grafana} and \textbf{OpenTelemetry} in Rust.}
        \item {Uses \textbf{Tonic} for gRPC, \textbf{Tokio} for async IO, and \textbf{Sqlite} with \textbf{S3} for storage.}
      \end{cvitems}
    }

  \cventry
    {\href{https://github.com/JosiahBull/gubber}{\faLink \space https://github.com/JosiahBull/gubber}}
    {Gubber}
    {}
    {} % TODO: date
    {
      \begin{cvitems}
        \item {A \textbf{Go} tool to automatically download and backup my \textbf{GitHub} repositories to my \textbf{Homelab}.}
        \item {Container using \textbf{Docker} and \textbf{Docker Compose}, backups are automatically diffed to reduce space.}
      \end{cvitems}
    }

  \cventry
    {\href{https://github.com/file-share-platform}{\faLink \space https://github.com/file-share-platform/}} % Empty position
    {RipTide}
    {}
    {2022}
    {
      \begin{cvitems} % Description(s) bullet points
        \item {A peer to peer file transfer utility with no attached file storage. Communication is over websockets and http.}
        \item {Built with \textbf{Rust} and \textbf{Vue}, in a team of two.}
      \end{cvitems}
    }

  \cventry
  {\href{https://github.com/JosiahBull/Convertium}{\faLink \space https://github.com/JosiahBull/Convertium}} % Empty position
  {Convertium}
  {}
  {2022}
  {
    \begin{cvitems} % Description(s) bullet points
      \item {A utility to automatically convert media files to reduce transcoding load when streaming with Plex.}
      \item {Powered by \textbf{Python}, \textbf{FFmpeg}, and \textbf{Docker}.}
    \end{cvitems}
  }

  % \cventry
  %   {\href{https://github.com/kemukupu}{\faLink \space https://github.com/kemukupu}} % Empty position
  %   {Kemu Kupu} % Project
  %   {} % Empty location
  %   {Oct. 2021} % Empty date
  %   {
  %     \begin{cvitems} % Description(s) bullet points
  %       \item {A fully complete spelling game for children to learn English and Maori words.}
  %       \item {Powered by \textbf{JavaFX}, \textbf{Warp}, and \textbf{Sqlite}, implemented in a team of three.}
  %     \end{cvitems}
  %   }

\end{cventries}

%   \cventry
%     {} % Empty position
%     {Zero Robotics} % Project
%     {} % Empty location
%     {} % Empty date
%     {
%       \begin{cvitems} % Description(s) bullet points
%       	\item {Semifinalist out of 200 teams in MIT's international high school programming competition in C.}
% 		\item {Implemented 3D vector physics and game strategy for an autonomous satellite simulation using the ZR API.}
%       \end{cvitems}
%     }
